% ------ datoteka: pr_uvod.tex preuzeta sa adrese http://poincare.matf.bg.ac.rs/~janicic//latex2e/

% ------ tip dokumenta je 'article' (npr. za radove, članke)
\documentclass{article}

% ------ pomoćni paket 'inputenc' za utf8 kodiranje
\usepackage[utf8]{inputenc}
\usepackage{wasysym}
% ------ početak teksta
\begin{document}
% ------ Poglavlje 1
\section{\TeX{} i \LaTeX}

\TeX{} je sistem za pripremu teksta za štampu koji je kasnih
sedamdesetih godina prošlog veka kreirao Donald Knut,
sa ciljem
da napravi \emph{sistem za obradu teksta 
namenjen pisanju lepih
knjiga, pogotovo knjiga koje sadrže puno matematike}.

\begin{center}
\LaTeX{} je sistem za pripremu za štampu nastao na osnovama
\TeX-a. \\ On definiše skup specifičnih \textbf{klasa
dokumenata} koje se koriste za formatiranje tekstova.
\end{center}

% ------ Poglavlje 2:
\section{Formatiranje teksta}

Pripremanjem teksta u \TeX-formatu potpuno precizno se
opisuje na koji će način on biti složen, pri čemu
su ti opisi najčešće savim prirodni i jednostavni.

\underline{Matematičke formule} se zapisuju izdvajanjem
pomoću simbola \$\$ na sledeći način:

$$\forall x \in S_{1}: x^{2} \geq \varepsilon $$

% ------ kraj teksta

\end{document}
