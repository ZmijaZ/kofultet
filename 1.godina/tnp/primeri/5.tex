\documentclass{article}
\usepackage[utf8]{inputenc}
\begin{document}
Strane u dokumentu se automatski numerišu, a eksplicitno
postavljanje rednog broja strane može se postići upotrebom
komande \linebreak[4] \verb|\setcounter{page}{broj}|. Komanda
\hspace{10mm} daje horizontalni razmak, a na\-red\-bom
\verb|\linebreak| možemo eksplicitno sugerisati prelamanje reda.
\bigskip
Doslovni zapis \framebox[2cm][r]{programa:}
\begin{minipage}[t]{60cm}
\begin{verbatim}
int sum(int n) {
  int i, s = 0;
  for(i=1; i<=n; i++) s=s+i;
  return s;
}
\end{verbatim}
\end{minipage}
\vspace{3\baselineskip}
\noindent Liste se kreiraju na sledeći način:
\begin{description}
\item[prva stavka] tekst...
\begin{itemize}
\item prva stavka
\item druga stavka
\end{itemize}
\item[druga stavka] tekst...
\begin{itemize}
\item[(a)] prva stavka
\item[(b)] druga stavka
\begin{itemize}
\item prva podstavka
\item druga podstavka
\end{itemize}
\end{itemize}
\end{description}
\end{document}