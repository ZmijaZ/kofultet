\documentclass{slides}

\begin{document}
	
	Cauchy-jev problem reda $n$ je definisan diferencijalnom
	jedna\v{c}inom:
	\begin{displaymath}
	\mathbf{u}’(x)=\mathbf{f}(x,\mathbf{u}(x))
	\end{displaymath}
	uz po\v{c}etni uslov $\mathbf{u}(x_{0})=\mathbf{u}^{(0
		)}$.
	Ovde je:
	\begin{displaymath}
	\mathbf{u}(x)=\left[
	\begin{array}{c}
	u_{0}(x) \\
	u_{1}(x) \\
	\ldots \\
	u_{n-1}(x)
	\end{array}
	\right]
	\end{displaymath}
	zatim:
	\begin{displaymath}
	\mathbf{f}(x,\mathbf{u}(x))=\left[
	\begin{array}{c}
	f_{0}(x,u_{0},u_{1},\ldots,u_{n-1}) \\
	f_{1}(x,u_{0},u_{1},\ldots,u_{n-1}) \\
	\ldots \\
	f_{n-1}(x,u_{0},u_{1},\ldots,u_{n-1})
	\end{array}
	\right]
	\end{displaymath}
	i:
	\begin{displaymath}
	\mathbf{u}^{(0)}=\left[
	\begin{array}{c}
	u_{0}^{(0)} \\
	u_{1}^{(0)} \\
	\ldots \\
	u_{n-1}^{(0)}
	\end{array}
	\right]
	\end{displaymath}
\end{document}